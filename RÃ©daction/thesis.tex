\documentclass{book}
\usepackage{thesisehess}

% Modifiez les informations ici ; pour Mme : M\up{me} 
\newcommand{\thesisauthorname}{Anouk MARTIN}
\newcommand{\thesistitle}{Titre du mémoire}
\newcommand{\thesissubtitle}{Le genre des arrangements économiques dans les familles recomposées}
\newcommand{\thesisdirectedby}{M\up{me} Sibylle Gollac, CNRS, CSU-CRESPPA}
\newcommand{\thesisdirectedbytwo}{M. Prénom Nom, établissement}
\newcommand{\thesisdefensedate}{le 5 juillet 2024}
\newcommand{\thesisrapporteurone}{M\up{me} Prénom Nom, établissement}
\newcommand{\thesisrapporteursecond}{M. Prénom Nom, établissement}
\newcommand{\thesisjuryone}{M\up{me} Sibylle Gollac, CNRS, CSU-CRESPPA}
\newcommand{\thesisjurytwo}{M\up{me} Cécile Brousse}
\newcommand{\thesisjurythree}{M. Prénom Nom, établissement}
\newcommand{\thesisjuryfour}{M\up{me} Prénom Nom, établissement}
\newcommand{\thesisjuryfive}{M. Prénom Nom, établissement}


\begin{document}


\makethesistitle


% Blankpage pour la page 2 :
\thispagestyle{empty} 


\chapter*{Remerciements}
\markboth{Remerciements}{Remerciements}
\addcontentsline{toc}{chapter}{Remerciements}
\blindtext


\chapter*{Résumé et mots clés}
\markboth{Résumé}{Résumé}
\addcontentsline{toc}{chapter}{Résumé}
\blindtext
\paragraph{Mots-clés} mot, mot, mot


\chapter*{Abstract and Keywords}
\markboth{Abstract and Keywords}{Abstract and Keywords}
\addcontentsline{toc}{chapter}{Abstract and Keywords}
\blindtext
\paragraph{Keywords} word, word, word


\tableofcontents
\markboth{Table des matières}{Table des matières}
\addcontentsline{toc}{chapter}{Table des matières}


\listoffigures
\markboth{Table des figures}{Table des figures}
\addcontentsline{toc}{chapter}{Table des figures}


\chapter*{Introduction}
\markboth{Introduction}{Introduction}
\addcontentsline{toc}{chapter}{Introduction}
\blindtext

% Pour les citations
\begin{quote}
    \onehalfspacing
\og{} \blindtext \fg{}
\end{quote}



% Les différents niveaux de titres
\part{Titre de la partie 1}
\chapter{Titre du chapitre}
\blindtext
\section*{Titre de niveau 1}
\blindtext
\subsection*{Titre de niveau 1.1}
\blindtext
\subsection*{Titre de niveau 1.1.1}
\blindtext


\chapter*{Bibliographie}
\markboth{Bibliographie}{Bibliographie}
\addcontentsline{toc}{chapter}{Bibliographie}

\begin{itemize}
    \item Ouvrages imprimés
    \item Ouvrages électroniques 
    \item Chapitre dans un ouvrage imprimé 
    \item Rapports imprimés 
    \item Travaux universitaires 
    \item Articles de périodiques imprimés 
    \item Articles de périodiques électroniques 
    \item Communication dans un congrès 
    \item Sites web consultés.
\end{itemize}


\chapter*{Annexe 1}
\markboth{Annexe 1}{Annexe 1}
\addcontentsline{toc}{chapter}{Bibliographie}
Chaque annexe doit commencer sur une nouvelle page et numérotée : Annexe 1 puis Annexe 2, etc.


\chapter*{Index géographique}
\markboth{Index géographique}{Index géographique}
\addcontentsline{toc}{chapter}{Index géographique}


\chapter*{Index général}
\markboth{Index général}{Index général}
\addcontentsline{toc}{chapter}{Index général}



\end{document}



